\documentclass[a4paper,11pt]{article}
\usepackage[T1]{fontenc}
\usepackage[utf8]{inputenc}
\usepackage{lmodern}
\usepackage{amsmath}
\usepackage{amssymb}

\title{Algorithms and Data Structures Coursework Questions 4-6}
\author{jsbl33}

\begin{document}

\maketitle

\section{Question 4}

a) $f(x) + g(x)$ is $o(f(x) \times g(x))$

For a function to be little-o of another function, $$\lim_{x \to \infty} \frac{a(x)}{b(x)} = 0$$\\

In our case,

\begin{gather*}
  \lim_{x \to \infty} \frac{f(x) + g(x)}{f(x)g(x)}\\
  = \lim_{x \to \infty} (\frac{f(x)}{f(x)g(x)} + \frac{g(x)}{f(x)g(x)})\\
  = \lim_{x \to \infty} (\frac{1}{g(x)} + \frac{1}{f(x)})\\
\end{gather*}

This is not always equal to zero, for instance when $f(x) = 1$ it is equal to 1.\\
$\therefore$ The statement is false.\\
\\
b) $2^x \times x^2$ is $o(2.1^x)$

Considering the limit as x tends to infinity

\begin{gather*}
  \lim_{x \to \infty} \frac{2^x \times x^2}{2.1^x}\\
  = \lim_{x \to \infty} (\frac{2}{2.1})^x \times x^2\\
  = 0
\end{gather*}

$\therefore$ The statement is true.\\
\\
c) $x^2 \times \log{x}$ is $O(x^2)$\\

For $x > 0$, $\log{x} < x$\\

$f(x) = x^2 \times \log{x} \leq x^2 \times x = x^3$\\

In order for the statement to be true, $x^3 \leq C \times x^2$ for a constant $C$ and $x > k$

$$x^3 \leq C \times x^2 \Rightarrow x \leq C$$\\

$\therefore C$ is not a constant so the statement is false.\\
\\
d) $x^2 \times \log{x}$ is $O(x^3)$\\

$$f(x) = x^2 \times \log{x}$$ For $x>0$, $\log{x} < x$\\

$\therefore f(x) = x^2 \times \log{x} \leq x^2 \times x = x^3$\\

For the statement to be true, $x^3 \leq C \times x^3$\\

$C = 1, k = 0$ is a valid witness pair that makes this true.\\

$\therefore$ The statement is true.\\
\\
e) $7x^5$ is $O(12x^4 + 5x^3 + 8)$

If $f(x) = O(12x^4 + 5x^3 + 8)$, then there must be a valid witness pair $C$ and $k$ such that\\

$7x^5 \leq C(12x^4 + 5x^3 + 8)$ for $x \geq k$\\

$\frac{7x^5}{12x^4 + 5x^3 + 8} \leq C$\\

$C$ is not a constant here so the statement is false.\\

\section{Question 5}

a) $T(n) = 64T(n/8) - n^2 \times \log{n}$\\

$\log_{b}{a} = 2, f(n) = n^2 \times \log{n}$\\

This is case 2 of the master theorem. $\therefore T(n) = \theta(n^2 \times \log{n})$\\
\\
b) $T(n) = 4T(n/2) + \frac{n}{\log{n}}$\\

$\log_{b}{a} = 2, f(n) = \frac{n}{\log{n}} = n \times \log^{-1}{n}$\\

This is case 1 of the master theorem, so $T(n) = \theta(n^2)$\\
\\
c) $T(n) = 2^n T(n/2) + n^n$\\

The master theorem cannot be applied here since coefficient of $T(n/2)$ is not a constant.\\
\\
d) $T(n) = 3T(n/4) + n \times \log{n}$\\

$\log_{b}{a} = 0.792, f(n) = n \times \log{n}$\\

This is looking like case 3 of the master theorem, though we need to check the regularity condition:\\

$\frac{3n}{4} \times \log{n/4} \leq cn\log{n}$\\

This is true for $c = 1$.\\

$\therefore T(n) = \theta(n \times log{n})$\\
\\
e) $T(n) = 3T(n/3) + n^{\frac{1}{2}}$\\

$\log_{b}{a} = 1, f(n) = n^{\frac{1}{2}}$\\

This is case 1 of the master theorem, $\therefore T(n) = \theta(n)$

\end{document}
